% ----------- 1. DECLARACIÓN DE CLASE Y OPCIONES -----------
\documentclass[acmsmall, screen=true, review=false]{acmart}

% 'acmsmall': Define el formato de columna simple para journals.
% 'screen=true': Habilita enlaces de colores para visualización en pantalla.
% 'review=false': Deshabilita el modo de revisión (numeración de líneas).

% ----------- 2. PAQUETES (El sistema ACM los carga automáticamente) -----------
% Si usas BibTeX, asegúrate de que use el estilo de citas de ACM (por defecto).
\citestyle{acmnumeric} % Opcional, pero define el estilo de citas (numérico).

% ----------- 3. METADATOS (TOP MATTER) -----------
\acmJournal{TOS} % Reemplaza con la abreviatura real de tu revista ACM
\acmVolume{1}
\acmNumber{1}
\acmArticle{1}
\acmYear{2025}
\acmMonth{11} % 11 para noviembre
\setcopyright{acmlicensed}

\title[Teoremas de Jerarquía]{Teoremas de Jerarquía}
\subtitle{Seccion 9.1 Indroducción a la Teoría de la Computación, Michael Sipser}

% Autor 1
\author{Ana Sofía Hernández Zavala}
%\orcid{319316717}
\affiliation{
  \institution{Universidad Nacional Autónoma de México, Facultad de Ciencias}
  \city{Ciudad de México}
  \country{México} % OBLIGATORIO
}
\email{anasofiahdzz@ciencias.unam.mx}

% Autor 2
\author{Nombre del Autor 2}
\affiliation{
  \institution{Otra Universidad/Departamento}
  \city{Otra Ciudad}
  \country{Otro País}
}
\email{autor2@ejemplo.com}

% Autor 3
\author{Nombre del Autor 3}
\affiliation{
  \institution{Otra Universidad/Departamento}
  \city{Otra Ciudad}
  \country{Otro País}
}
\email{autor2@ejemplo.com}

%checar esto
% Conceptos de Clasificación de Computación ACM (CCS)
% Obligatorio para artículos > 2 páginas.
\begin{CCSXML}
<ccs2012>
 <concept>
  <concept_id>10003033.10003083.10003095</concept_id>
  <concept_desc>Networks~Network reliability</concept_desc>
  <concept_significance>500</concept_significance>
 </concept>
</ccs2012>
\end{CCSXML}

\ccsdesc[500]{Networks~Network reliability}

% Palabras clave (Keywords)
\keywords{teoremas, jerarquías, tiempo, espacio, corolario, definicion, máquina de Turing}

% ----------- 4. COMIENZO DEL DOCUMENTO -----------
\begin{document}

% 1. EL RESUMEN VA PRIMERO, DENTRO DE begin{document}
\begin{abstract}
Este es el resumen de la investigación. (Ya puedes borrar el comentario confuso que estaba aquí).
\end{abstract}

% 2. LUEGO SE GENERA EL BLOQUE DE TÍTULO Y METADATOS
\maketitle

% 3. LUEGO VIENE LA PRIMERA SECCIÓN
\section{Introducción}
El sentido común nos dice que si le damos más tiempo o más espacio a una máquina de Turing entonces debería de incrementar la classe de problemas que podría resolver; y los Teoremas de Jerarquía lo confirman, ya que estos teoremas prueban que las clases de complejidad de tiempo y espacio no son todas las mismas. \\
Por ejemplo en este articulo mostraremos que el teorema de jerarquía de complejidad del espacio es más simple que el del tiempo.

\section{Teorema del Espacio}
\begin{definition}
    Una función $f: N \rightarrow N $, donde $f(n)$ es al menos $O(log_n)$, es llamada espacio constructible si la funcion que mapea la cadena de $1^n$ a la representación binaria de $f(n)$ y es computada en espacio $O(f(n))$.
\end{definition}

 \cite{cita1, cita2}.

\subsection{Resultados}
Se encontró una mejora significativa.

\section{Conclusiones}
Resumimos nuestros hallazgos aquí.

% (Opcional) Agradecimientos: se recomienda usar el entorno 'acks'
\begin{acks}
  Agradecemos a la Fundación de Ejemplo por el apoyo financiero (Grant \grantnum{GS99999}{123456}).
\end{acks}

% ----------- 5. BIBLIOGRAFÍA -----------
\bibliographystyle{ACM-Reference-Format}
\bibliography{referencias} % Llama al archivo referencias.bib

\end{document}